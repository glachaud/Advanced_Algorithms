\documentclass[11pt]{article}
\usepackage{amsmath}
\usepackage{amsfonts}
\usepackage{mathtools}
\usepackage{pdfpages}
\usepackage{color}
\usepackage{fullpage}
\usepackage{listings}
\usepackage[french, english]{babel}
\usepackage{stmaryrd}
\usepackage{fancyhdr}

\usepackage[simplified]{pgf-umlcd}
\usepackage{tikz}

\usepackage{hyperref}
\hypersetup{
    colorlinks=true,
    linkcolor=blue,
    filecolor=magenta,      
    urlcolor=cyan,
}


\lstdefinestyle{customjava}{
  belowcaptionskip=1\baselineskip,
  breaklines=true,
  xleftmargin=\parindent,
  language=Java,
  numbers=left,
  showstringspaces=false,
  basicstyle=\footnotesize\ttfamily,
  keywordstyle=\bfseries\color{green!40!black},
  commentstyle=\itshape\color{purple!40!black},
  identifierstyle=\color{blue},
  stringstyle=\color{orange},
}

\allowdisplaybreaks
\setlength{\headsep}{2em}
\pagestyle{fancy}
\fancyhf{}
\lhead{TP3: Graphs}
\rhead{Guillaume Lachaud}
\rfoot{\thepage}
\begin{document}
\begin{titlepage}
    \centering
    \vfill
    {\bfseries\Huge
       	Tutorial course 3: Graphs, an introduction\\
        \vskip 1em
        \hrule
        \Large
        \vskip2cm
        \vskip1em
        Guillaume Lachaud
    }    
    \vfill
    % \includegraphics[scale=1]{isep_logo}
    \vfill
    \vfill
\end{titlepage}
\tableofcontents
\clearpage
\listoffigures
\clearpage

\section*{Manipulating graph data}
\addcontentsline{toc}{section}{Manipulating graph data}

\subsection*{Adjacency list representation of an undirected graph}
\addcontentsline{toc}{subsection}{Adjacency list representation of an undirected
graph}

Some liberty has been taken in the implementation of the adjacency list. This
has been done in order for the representation to be as useful as possible.

\subsection*{Adjacency matrix representation of an undirected graph}
\addcontentsline{toc}{Subsection}{Adjacency matrix representation of an
undirected graph}

As in the case of the adjacency list, some liberty has been taken in the
implementation. 

\section*{Practical application}
\addcontentsline{toc}{section}{Practical application}

Node 1 has 16 connections and Node 34 has 17 connections. They are by far the
two nodes with the highest number of connections.
\end{document}
